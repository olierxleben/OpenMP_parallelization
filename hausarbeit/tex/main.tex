\documentclass{lni}

\IfFileExists{latin1.sty}{\usepackage{latin1}}{\usepackage{isolatin1}}

\usepackage{graphicx}
% use of 
\usepackage{minitoc}

% add bibliography to toc
\usepackage[nottoc]{tocbibind}

\begin{document}
\author{
	Martin Helmich, Oliver Erxleben \\ 
	\\ 
	Hochschule Osnabrück \\ 
	Ingenieurswissenschaften und Informatik \\ 
	Informatik - Mobile und Verteilte Anwendungen \\ 
	\\ 
	martin.helmich@hs-osnabrueck.de \\
	oliver.erxleben@hs-osnabrueck.de
}

\title{OpenMP [Working DRAFT]}

\maketitle

\tableofcontents

\pagebreak % new page after toc


\begin{abstract}
Die vorliegende Hausarbeit wurde in LateX verfasst und ist eine gemeinsame Arbeit von Martin Helmich und Oliver Erxleben. Das Thema der Arbeit ist die konzeptionelle und technische Betrachtung des Einsatzes von OpenMP zur Entwicklung von parallelen Programmen. TODO: Cliffhanger
\end{abstract}

\section{OpenMP: Konzept}

Dies ist ein Kapitel. Text Text Text Text Text Text Text Text Text Text Text Text Text Text Text Text Text Text Text Text Text Text Text Text Text Text Text Text Text Text Text Text Text Text Text Text Text Text Text Text Text Text Text Text Text Text Text Text Text Text Text Text Text Text Text Text Text Text Text Text Text Text Text.

Text Text Text Text Text Text Text Text Text Text Text Text Text Text Text Text Text Text Text Text Text Text Text Text Text Text Text Text Text Text Text Text Text Text Text Text Text Text Text Text Text Text Text Text Text Text Text Text Text.

\subsection{Join-Fork-Modell}
\subsubsection{Programmbeispiel}
\subsection{Vor- und Nachteile}

\pagebreak

\section{Vergleich mit TBB und anderen?!}

Text Text Text Text Text Text Text Text Text Text Text Text Text Text Text Text Text Text Text Text Text Text Text Text Text Text Text Text Text Text Text Text Text Text Text Text Text Text Text Text Text Text Text Text Text Text Text Text Text Text Text Text Text Text Text Text.

\subsection{Unterkapitelüberschrift}

Text Text Text Text Text Text Text Text Text Text Text Text Text Text Text Text Text Text Text Text Text Text Text Text Text Text Text Text Text Text Text Text Text Text Text Text Text Text.

%\begin{figure}[htb]
%  \begin{center}
%    \includegraphics[width=2cm]{gilogo}
%    \caption{\label{logo}Beschreibung der Abbildung}
%  \end{center}
%\end{figure}

Text Text Text Text Text Text Text Text Text Text Text Text Text Text Text Text Text Text Text Text Text Text Text Text Text Text Text Text Text Text Text Text Text Text Text Text Text Text.

Text mit Fußnote.\footnote{Dies ist eine Fußnote. Text Text Text Text Text Text Text Text Text Text Text Text Text Text Text Text Text Text Text.} Text \cite{Ez99,ABC01} Text Text Text Text Text Text Text Text Text Text Text Text Text Text Text Text Text Text Text Text Text Text Text Text Text Text Text Text  Text Text Text Text Text Text Text Text Text Text Text Text Text Text Text Text Text Text Text Text Text Text Text Text Text Text Text Text Text Text Text Text Text Text Text Text Text Text Text Text Text.

\pagebreak
\bibliographystyle{plain}
\bibliography{lib/lnitemplate}

\newpage

\listoffigures

\listoftables

\section{Anlagen}
\pagenumbering{none}
\setcounter{subsection}{0} %clear counter 
\minitoc

\nomtcpagenumbers % TODO: the dots remain in eternity... 

\subsection{Source Code}

\subsection{Latex Dateien}


% Vor Abgabe raus!!!! 
\subsection{Working Notes}
Links: 
https://computing.llnl.gov/tutorials/openMP/ \\
http://www.people.westminstercollege.edu/faculty/ggagne/fall2012/306/handouts/parallel/index.html \\
http://www.openmp.org/mp-documents/OpenMP3.1.pdf \\

\end{document}
