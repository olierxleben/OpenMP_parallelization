\documentclass{lni}

\IfFileExists{latin1.sty}{\usepackage{latin1}}{\usepackage{isolatin1}}

\usepackage{graphicx}
\usepackage{minitoc}

% add bibliography to toc
\usepackage[nottoc]{tocbibind}

\begin{document}

\author{
	Martin Helmich, Oliver Erxleben \\ 
	\\ 
	Hochschule Osnabrück \\ 
	Ingenieurswissenschaften und Informatik \\ 
	Informatik - Mobile und Verteilte Anwendungen \\ 
	\\ 
	martin.helmich@hs-osnabrueck.de \\
	oliver.erxleben@hs-osnabrueck.de
}

\title{\includegraphics[scale=0.75,keepaspectratio]{img/hs_os.png}\linebreak \linebreak OpenMP [Working DRAFT]}

\maketitle

\tableofcontents

\begin{abstract}
Die vorliegende Hausarbeit ist eine gemeinsame Arbeit von Martin Helmich und Oliver Erxleben für das Modul \textit{Parallele und Verteilte Algorithmen} im Wintersemester 2012/13 an der Hochschule Osnabrück. Das Thema der Arbeit Betrachtung von OpenMP zur Parallelisierung von Programmen. \\ 
Die Arbeit gliedert sich in die konzeptionelle Betrachtung und kritische Würdigung von OpenMP und den Vergleich mit der im Modul kennengelernten Programmier- \\ bibliothek Threading Building Blocks von Intel. \\
Desweiteren wird die Aufgabe aus Intel`s \textit{Accelerate your Code} - Programmierwettbewerb von Herbst 2012 mittels OpenMP reimplementiert.
\end{abstract}

\pagebreak % new page after toc and abstract

\setcounter{page}{1} % the real document begins here, so set page number to 1
\section{OpenMP: Konzept}

\subsection{Join-Fork-Modell}
\subsubsection{Programmbeispiel}
\subsection{Vor- und Nachteile}

\section{Vergleich mit TBB und anderen?!}

Text Text Text Text Text Text Text Text Text Text Text Text Text Text Text Text Text Text Text Text Text Text Text Text Text Text Text Text Text Text Text Text Text Text Text Text Text Text Text Text Text Text Text Text Text Text Text Text Text Text Text Text Text Text Text Text.

\subsection{Unterkapitelüberschrift}

Text Text Text Text Text Text Text Text Text Text Text Text Text Text Text Text Text Text Text Text Text Text Text Text Text Text Text Text Text Text Text Text Text Text Text Text Text Text.

%\begin{figure}[htb]
%  \begin{center}
%    \includegraphics[width=2cm]{gilogo}
%    \caption{\label{logo}Beschreibung der Abbildung}
%  \end{center}
%\end{figure}

Text Text Text Text Text Text Text Text Text Text Text Text Text Text Text Text Text Text Text Text Text Text Text Text Text Text Text Text Text Text Text Text Text Text Text Text Text Text.

Text mit Fußnote.\footnote{Dies ist eine Fußnote. Text Text Text Text Text Text Text Text Text Text Text Text Text Text Text Text Text Text Text.} Text \cite{Ez99,ABC01} Text Text Text Text Text Text Text Text Text Text Text Text Text Text Text Text Text Text Text Text Text Text Text Text Text Text Text Text  Text Text Text Text Text Text Text Text Text Text Text Text Text Text Text Text Text Text Text Text Text Text Text Text Text Text Text Text Text Text Text Text Text Text Text Text Text Text Text Text Text.

\pagebreak

\bibliography{lib/references}{}
\bibliographystyle{plain}

\newpage

\listoffigures

\listoftables

\section{Anlagen}
\pagenumbering{none}
\setcounter{subsection}{0} %clear counter 
\minitoc

\nomtcpagenumbers % TODO: the dots remain in eternity... 

\subsection{Source Code}

\subsection{Latex Dateien}


% Vor Abgabe raus!!!! 
\subsection{Working Notes}
Links: 
https://computing.llnl.gov/tutorials/openMP/ \\
http://www.people.westminstercollege.edu/faculty/ggagne/fall2012/306/handouts/parallel/index.html \\
http://www.openmp.org/mp-documents/OpenMP3.1.pdf \\

\end{document}
	